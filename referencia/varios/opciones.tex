
\newdefinition{OPCST}{opción de estilo}{opciones de estilo}{Son los valores que modifican el funcionamiento del estilo. Se ponen entre corchetes y separadas por comas en el comando \textbackslash documentclass y antes de el nombre del estilo que irá entre llaves}

En este apéndice se presentan las múltiples \dfnpl{OPCST}\PhIndex{opciones} (clase) y sus funciones de forma resumida aunque todas ellas ya han sido presentadas donde corresponde es aconsejable disponer de un resumen de ellas. Estas opciones son las siguientes:

\begin{itemize}
  \item Tamaño de la página
  \begin{description}
    \item [normalbook] Opción por defecto, no hace falta indicarla. Utiliza A4 como tamaño de página.
    \item [smallbook] Con esta opción se utiliza el tamaño B5 como tamaño de página y se reduce la tipografía de acuerdo con la reducción de tamaño.
    \item [tinybook]  Con esta opción se utiliza el tamaño C5 como tamaño de página, siendo el tamaño más pequeño posible y se reduce la tipografía de acuerdo con la reducción de tamaño.
  \end{description}
  \item Idioma
  \begin{description}
    \item [spanish] Opción por defecto, no hace falta indicarla. Usa el español como lenguaje base. El abstract y el resumen se ordenan de acuerdo con ello.
    \item [english] Usa el idioma inglés como lenguaje base. El abstract y el resumen se ordenan de acuerdo con ello.
  \end{description}
  \item Tipo de documento
  \begin{description}
    \item [tfg] Opción por defecto, no hace falta indicarla. El documento será un trabajo fin de grado.
    \item [tfm] El documento será un trabajo fin de máster.
    \item [thesis] El documento será una tesis.
  \end{description}
  \item Tipo de copyright
  \begin{description}
    \item [copyright] Opción por defecto, no hace falta indicarla. Muestra el copyright normal en el reverso de la portada.
    \item [copyleft] Muestra el copyleft en el reverso de la portada.
    \item [nocopyright] No muestra ni copyright ni copyleft en el reverso de la portada.
  \end{description}
  \item Tipo de índice.
  \begin{description}
    \item [normalindex] Opción por defecto, no hace falta indicarla. Muestra partes, capítulos y apartados.
    \item [extendedindex] Muestra también los subapartados. Es aconsejable pesnsar si en el documento creado tiene sentido o no mostrar ese nivel.
    \item [fullindex] Muestra hasta el nivel de subsubapartados. Su uso no se aconseja pero está disponible por si en algún caso se considera necesario dada la extensión del documento y la necesidad de mostrar en el índice la organización del documento hasta ese nivel.
  \end{description}
  \item Listados de figuras, ecuaciones, algoritmos ...
  \begin{description}
    \item [loall] Muestra todos los listados.
    \item [nonelo] No muestra ningún listado.
    \item [loa] Muestra el listado de algoritmos y es compatible con cualquier otra opción menos con loall y nonelo. Se recomienda que no se utilice si el número de algoritmos en todo el documento es menor de tres salvo que se considere adecuado hacerlo.
    \item [loc] Muestra el listado de códigos y es compatible con cualquier otra opción menos con loall y nonelo. Se recomienda que no se utilice si el número de códigos en todo el documento es menor de tres salvo que se considere adecuado hacerlo.
    \item [loe] Muestra el listado de ecuaciones y es compatible con cualquier otra opción menos con loall y nonelo. Se recomienda que no se utilice si el número de ecuaciones en todo el documento es menor de tres salvo que se considere adecuado hacerlo.
    \item [lof] Muestra el listado de figuras y es compatible con cualquier otra opción menos con loall y nonelo. Se recomienda que no se utilice si el número de figuras en todo el documento es menor de tres salvo que se considere adecuado hacerlo.
    \item [lot] Muestra el listado de tablas y es compatible con cualquier otra opción menos con loall y nonelo. Se recomienda que no se utilice si el número de tablas en todo el documento es menor de tres salvo que se considere adecuado hacerlo.
    \item [lotb] Muestra el listado de cuadros de texto y es compatible con cualquier otra opción menos con loall y nonelo. Se recomienda que no se utilice si el número de cuadros de texto en todo el documento es menor de tres salvo que se considere adecuado hacerlo.
  \end{description}
  \item Aspecto global
  \begin{description}
    \item [covers] Presenta las cubiertas institucionales.
    \item [final] Se prepara para la versión final eliminando la marca de agua.
    \item [printable] Por defecto se compila el documento con márgenes simétrico y esta opción lo que hace es añadir margen en la parte derecha de las páginas pares y en la izquierda de las páginas impares para dejar espacio para la encuadernación. Se recomienda una compilación con esta opción para la entrega electrónica y otra con ella para la entrega en papel.
    \item [firstnumbered] Por defecto las primeras páginas de cada capítulo no se numeran. Si se desea que sean numeradas debe ponerse esta opción.
  \end{description}
  \item Imagen institucional
  \begin{description}
    \item [epsbased] Dado que este estilo ha sido diseñado inicialmente es la opción por defecto y no hace falta indicarla. Utiliza los colores institucionales de la Escuela Politécnica Superior y los logos y textos adecuados.
    \item [uambased] Es una opción genérica que utiliza los colores institucionales de la Universidad Autónoma de Madrid. Es necesario indicar el logo, su ancho, la facultad y cuantas variables sean necesarias para el correcto formato del documento.
    \item [cienciasbased] Utiliza los colores institucionales de la Facultad de Ciencias y los logos y textos adecuados.
    \item [economicasbased] Utiliza los colores institucionales de la Facultad de Ciencias Económicas y Empresariales y los logos y textos adecuados.
    \item [derechobased] Utiliza los colores institucionales de la Facultad de Derecho y los logos y textos adecuados.
    \item [filosofiabased] Utiliza los colores institucionales de la Facultad de Filosofía y Letras y los logos y textos adecuados.
    \item [medicinabased] Utiliza los colores institucionales de la Facultad de Medicina y los logos y textos adecuados.
    \item [enfermeriabased] Utiliza los colores institucionales de las Escuelas de Enfermería, sin embargo al haber varias y los logos y textos deberán ser modificados utilizando las funciones adecuadas.
    \item [fisioterapiabased] Utiliza los colores institucionales de las Escuelas de Fisioterapia, sin embargo al haber varias y los logos y textos deberán ser modificados utilizando las funciones adecuadas.
  \end{description}
  \item Colores globales. Dado que sólo controlan los colores del documento es necesario utilizar todos los comandos necesarios para indicar los textos y logos del documento.
    \begin{description}
      \item [blackbased] Todas las decoraciones estarán en negro.
      \item [graybased] Todas las decoraciones estarán en tonos de gris.
      \item [redbased] Todas las decoraciones estarán en rojo.
      \item [greenbased] Todas las decoraciones estarán en verde.
      \item [bluebased] Todas las decoraciones estarán en azul.
      \item [yellowbased] Todas las decoraciones estarán en amarillo.
      \item [magentabased] Todas las decoraciones estarán en magenta.
      \item [cyanbased] Todas las decoraciones estarán en cian.
      \item [orangebased] Todas las decoraciones estarán en naranja.
    \end{description}
    \item Control de compilación
      \begin{description}
        \item [overleaf] Permite compilar en overleaf pero no se dispone de xindy que es quien permite una ordenación de los acrónimos y las definiciones teniendo en cuenta que las letras con acento se ordenan con las letras sin acento. No se podrá utilizar tampoco la opción gnuplot.
        \item [gnuplot] Permite compilar código gnuplot.
      \end{description}
\end{itemize}
