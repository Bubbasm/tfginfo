La predicción del tráfico es un componente esencial de la planificación, el desarrollo y la gestión de redes. 
En un mundo cada vez más conectado, donde la demanda de servicios y aplicaciones móviles continúa creciendo, 
es fundamental poder anticipar y adaptarse a los cambios en el tráfico de datos.
La predicción del tráfico se refiere a la capacidad de estimar la cantidad de datos que se transmitirán 
a través de una red en un período de tiempo determinado. 
Esto es crucial para garantizar un rendimiento óptimo de la red, evitar congestiones y 
planificar eficientemente la asignación de recursos.

Es importante considerar también los posibles ataques en la red, como los Ataques de \ac{DOS}
y los Ataques \ac{DDOS}. Estos ataques pueden afectar significativamente 
la capacidad de la red para operar normalmente, sobrecargando los recursos y causando interrupciones en el servicio. 
La predicción del tráfico puede ayudar a identificar patrones asociados con estos ataques 
y permitir respuestas proactivas para mitigar su impacto.

En este trabajo se explorarán técnicas de predicción del tráfico estudiadas otros trabajos,
como el método de Holt-Winters \cite{cabornero2021}, o el ajuste a distribuciones $\alpha$-estables \cite{benjamin2023}.
Además, se investigarán técnicas de estacionalización mediante diferenciación, y su normalización, para
proponer una metodología simple que permita detectar anomalías en el tráfico de red.
