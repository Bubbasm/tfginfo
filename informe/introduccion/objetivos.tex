El objetivo de este trabajo es dar continuidad a las investigaciones realizadas por Benjamín en su
\ac{TFG} titulado ``Detección de ciberataques de denegación de servicio mediante Funciones Características'' \cite{benjamin2021}
y en su \ac{TFM} titulado ``Estudio de la predictibilidad del tráfico en Internet para la detección de anomalías sutiles'' \cite{benjamin2023}.

Decir que queremos mejorar la precisino de detección de ataques de anomalías sutiles en la red
basándonos únicamente en la cantidad de datos brutos que están circulando por la red. Este estudio
ofrece una forma de detectar anomalías simples, sin necesidad de procesar muchos datos o de
tener que filtrarlos previamente.
Para ello, se utilizarán dos series temporales que contienen información
sobre el ancho de banda en un enlace específico: una serie sin ataques (denominada como tráfico normal)
y otra serie creada aplicando un ataque al tráfico normal (tráfico de ataque).

El enfoque de este estudio consiste en detectar anomalías en la serie temporal del tráfico de red.
Para lograr esto, se aplicarán modelos predictivos para series temporales, 
como el método de Holt-Winters. También se estudiarán técnicas de estacionalización mediante diferenciación y normalización,
ajustes a distribuciones $\alpha$-estables y la descomposición de la serie temporal basada en su estacionalidad.