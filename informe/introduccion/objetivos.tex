El objetivo de este trabajo es dar continuidad a las investigaciones realizadas por Benjamín Martín en su
\ac{TFG} titulado ``Detección de ciberataques de denegación de servicio mediante Funciones Características''~\cite{benjamin2021}
y en su \ac{TFM} titulado ``Estudio de la predictibilidad del tráfico en Internet para la detección de anomalías sutiles''~\cite{benjamin2023}.

Para mejorar la precisión en la detección de ataques de anomalías sutiles en la red, nuestro enfoque se centra en analizar la cantidad de datos brutos circulando por la red. Este estudio propone un método simple para detectar anomalías. Se emplean dos series temporales que describen el ancho de banda en un enlace específico: una serie representa el tráfico normal y la otra simula un ataque aplicado al tráfico normal.
Este trabajo busca identificar y diferenciar patrones anómalos con un procesamiento eficiente, sin la necesidad de un filtrado exhaustivo o un volumen masivo de datos.

El enfoque de este estudio consiste en detectar anomalías en la serie temporal del tráfico de red.
Para lograr esto, se aplicarán modelos predictivos para series temporales, 
como el método de Holt-Winters. También se estudiarán técnicas de estacionarización mediante diferenciación,
ajustes a distribuciones $\alpha$-estables y la descomposición de la serie temporal basada en su estacionalidad.