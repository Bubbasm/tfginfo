Por las características de los datos que hemos observado, se plantea que la combinación de ambos métodos puede ser útil para la detección de anomalías.
Por ello, se propone un análisis de regresión logística y \ac{SVM}, utilizando como variables independientes el parámetro \textit{mean} y $\delta$ obtenidos en el ajuste de la distribución $\alpha$-estable. SVM es útil para separar eficazmente datos en diferentes clases mediante la identificación de un hiperplano óptimo en un espacio dimensional superior.

\begin{figure}[SVM de los parámetros $\delta$ y \textit{mean}, utilizando kernel RBF]{FIG:SVMPLOT}{SVM de los parámetros $\delta$ y \textit{mean}, utilizando kernel RBF}
    \image{6cm}{}{svm_delta_mean.pdf}
    \label{FIG:SVMPLOT}
\end{figure}

\begin{figure}[Matriz de confusión y curva ROC del SVM de los parámetros $\delta$ y \textit{mean}]{FIG:CONFUSIONMATRIXSVM}{Matriz de confusión y curva ROC del SVM de los parámetros $\delta$ y \textit{mean}}
    \subfigure[SFIG:CONFUSIONSVM]{Matriz de confusión}{
        \image{7cm}{}{confusion_matrix_svm.pdf}
    } \quad
    \subfigure[SFIG:ROCSVM]{Curva ROC}{
        \image{7cm}{}{roc_curve_svm.pdf}
    }
\end{figure}

En SVM, el \textit{kernel} \ac{RBF} se utiliza comúnmente para abordar problemas de clasificación no lineales, ya que permite capturar relaciones más complejas entre las características de entrada. En la figura~\ref{FIG:SVMPLOT} se muestra el resultado de la clasificación de los parámetros $\delta$ y \textit{mean} utilizando un \textit{kernel} RBF. 
Se observa que la clasificación obtenida sigue un patrón lineal, por lo que se plantea que un \textit{kernel} lineal puede ser suficiente para clasificar los datos. En este caso, la regresión logística puede ser una alternativa más sencilla y eficiente (y en algunos casos mejor~\cite{salazar2012comparison}) para clasificar los datos.

Aplicando SVM, se ha obtenido una exactitud o \textit{accuracy} del $70.05\%$, una precisión del $70.09\%$, una sensibilidad o \textit{recall} del $67.98\%$ y un \textit{F1 score} del $69.41\%$. En al figura~\ref{FIG:CONFUSIONMATRIXSVM} observamos la matriz de confusión del SVM con los parámetros $\delta$ y \textit{mean} y su curva \ac{ROC}.

\begin{table}[Resultados de la regresión logística con parámetros $\delta$ y \textit{mean}]{TAB:LOGISTICREGRESSION}{Resultados de la regresión logística con parámetros $\delta$ y \textit{mean}}
    \begin{tabular}{|c|c|c|c|c|}
        \hline
        & coef & std err & z & P>|z| \\
        \hline
        const & $-1.0319$ & $0.018$ & $-57.341$ & $0.000$ \\
        $\delta$ & $-0.0010$ & $0.002$ & $-0.644$ & $0.519$ \\
        mean & $62.9153$ & $0.744$ & $84.588$ & $0.000$ \\
        \hline
    \end{tabular}
\end{table}

En la tabla~\ref{TAB:LOGISTICREGRESSION} se muestran los resultados de la regresión logística. Para poner a prueba la eficacia de la regresión logística, se ha realizado una validación cruzada con un $80\%$ de los datos para entrenamiento y un $20\%$ para validación. Se ha procurado que las ventanas que se usan para entrenamiento y \textit{test} no se solapen en ningún punto, ya que esto podría introducir sesgos en el modelo. Esto se ha conseguido eliminando los datos de validación que compartan algún punto con los datos de entrenamiento. 

\begin{figure}[Matriz de confusión y curva ROC de la regresión logística con parámetros $\delta$ y \textit{mean}]{FIG:CONFUSIONMATRIX1}{Matriz de confusión y curva ROC de la regresión logística con parámetros $\delta$ y \textit{mean}}
    \subfigure[SFIG:CONFUSION1]{Matriz de confusión}{
        \image{7cm}{}{confusion_matrix_1.pdf}
    } \quad
    \subfigure[SFIG:ROC1]{Curva ROC}{
        \image{7cm}{}{roc_curve_1.pdf}
    }
\end{figure}


En la figura~\ref{FIG:CONFUSIONMATRIX1} se muestra la matriz de confusión de la regresión logística con los parámetros $\delta$ y \textit{mean} y su curva \ac{ROC}.
Se ha obtenido un \textit{accuracy} del $79.4\%$, una precisión del $79.8\%$, un \textit{recall} del $78.7\%$ y un \textit{F1 score} del $79.3\%$. Estos resultados indican que la regresión logística es capaz de clasificar correctamente la mayoría de los datos, aunque se observa que hay un número significativo de falsos positivos y falsos negativos. Por ello, se plantea que la regresión logística puede ser útil para clasificar los datos, pero no es suficiente para obtener una clasificación perfecta.

\begin{table}[Resultados de la regresión logística con todos los parámetros]{TAB:LOGISTICREGRESSIONFULL}{Resultados de la regresión logística con todos los parámetros}
    \begin{tabular}{|c|c|c|c|c|}
        \hline
        & coef & std err & z & P>|z| \\
        \hline
        const & $-0.1488$ & $0.122$ & $-1.218$ & $0.223$ \\
        $\alpha$ & $-0.4842$ & $0.069$ & $-7.064$ & $0.000$ \\
        $\beta$ & $-0.3669$ & $0.039$ & $-9.426$ & $0.000$ \\
        $\gamma$ & $0.4641$ & $0.065$ & $7.130$ & $0.000$ \\
        $\delta$ & $-0.0020$ & $0.002$ & $-1.003$ & $0.316$ \\
        mean & $64.9292$ & $0.769$ & $84.453$ & $0.000$ \\
        std & $-0.6110$ & $0.051$ & $-11.943$ & $0.000$ \\
        \hline
    \end{tabular}
\end{table}

\begin{figure}[Matriz de confusión y curva ROC de la regresión logística con todos los parámetros]{FIG:CONFUSIONMATRIX2}{Matriz de confusión y curva ROC de la regresión logística con todos los parámetros}
    \subfigure[SFIG:CONFUSION2]{Matriz de confusión}{
        \image{7cm}{}{confusion_matrix_2.pdf}
    } \quad
    \subfigure[SFIG:ROC2]{Curva ROC}{
        \image{7cm}{}{roc_curve_2.pdf}
    }
\end{figure}

También se ha realizado un análisis de regresión logística con todos los parámetros considerados. Los resultados de la regresión logística se muestran en la tabla~\ref{TAB:LOGISTICREGRESSIONFULL}.
En la figura~\ref{FIG:CONFUSIONMATRIX2} se muestra la matriz de confusión de la regresión logística con todos los parámetros y su curva \ac{ROC}. Se ha obtenido un \textit{accuracy} del $83.1\%$, una precisión del $82.2\%$, una sensibilidad del $84.6\%$ y un \textit{F1 score} del $83.4\%$. Estos resultados indican que la regresión logística con todos los parámetros es más eficaz que la regresión logística que solo usa los parámetros $\delta$ y \textit{mean}, ya que se obtiene una mejora en todas las métricas de evaluación.

\subsection{Detecci\'on sobre el conjunto de datos de prueba}\label{SUBSEC:DETECCIONDATASET}

Un resultado de interés es comprobar en qué ventanas del \textit{dataset} se pueden detectar anomalías y en qué ventanas se puede confirmar que no hay anomalías correctamente. Para ello, se ha realizado una predicción utilizando el método de regresión logística con todos los parámetros, ya que obtenemos los mejores resultados con este método. Se ha utilizado el $80\%$ de los datos para el entrenamiento de la regresión, y se ha realizado la prueba sobre los datos restantes, tras eliminar los datos que puedan solaparse con los datos de entrenamiento.

\begin{figure}[Clasificación de las ventanas del \textit{dataset} sujeto a ataques, entre los días 09/06/2016 (jueves) y 10/06/2016 (viernes), utilizando regresión logística con todos los parámetros]{FIG:CATALOGACIONATT}{Clasificación de las ventanas del \textit{dataset} sujeto a ataques, entre los días 09/06/2016 (jueves) y 10/06/2016 (viernes), utilizando regresión logística con todos los parámetros}
    \image{}{}{catalogacion_ataque.pdf}
\end{figure}

En la figura~\ref{FIG:CATALOGACIONATT} se muestra la clasificación de las ventanas del \textit{dataset} utilizando regresión logística con todos los parámetros. Para una representación más clara, se ha utilizado una ventana móvil de 60 segundos.
En rojo se señalan las ventanas incorrectamente clasificadas como no ataques, mientras que en verde se destacan las ventanas correctamente clasificadas como ataques. La mayoría de las ventanas están clasificadas correctamente, pero hay un número considerable de falsos negativos que tienden a concentrarse alrededor de ciertas áreas. El porcentaje de falsos negativos es del $21.87\%$. Se aprecia que los falsos negativos suceden en ventanas donde hay mucha variabilidad del tráfico. En este caso sucede en los periodos entre las 09:00 y 11:00 y también entre las 16:00 y 19:00 del jueves, 09, de junio de 2016. Estas clasificaciones incorrectas se deben a que el tráfico en estos periodos se asemeja al tráfico de ataque, mostrando una carga incremental, por lo que el modelo presenta más dificultades para clasificar correctamente estas ventanas.

\begin{figure}[Clasificación de las ventanas del \textit{dataset} sin aplicar ataque, entre los días 09/06/2016 (jueves) y 10/06/2016 (viernes), utilizando regresión logística con todos los parámetros]{FIG:CATALOGACIONDEF}{Clasificación de las ventanas del \textit{dataset} sin aplicar ataque, entre los días 09/06/2016 (jueves) y 10/06/2016 (viernes), utilizando regresión logística con todos los parámetros}
    \image{}{}{catalogacion_defensa.pdf}
    \label{FIG:CATALOGACIONDEF}
\end{figure}

En la figura~\ref{FIG:CATALOGACIONDEF} se muestra la clasificación de las ventanas del dataset utilizando regresión logística con todos los parámetros. Se ha utilizado una ventana móvil de 60 segundos para una representación más clara.
En color rojo se indican las ventanas que se han incorrectamente clasificadas como ataque, mientras que en color azul se indican las ventanas que han sido correctamente clasificadas como no ataque.
Al igual que en el caso anterior, la mayoría de las ventanas han sido clasificadas correctamente, y el porcentaje de falsos positivos es del $17.24\%$.
En este caso, no se observan concentraciones de falsos positivos, sino que se distribuyen de forma más uniforme a lo largo del día, por lo que podemos concluir que el método no está influenciado por ventanas con mucho o poco tráfico.
