Como se ha encontrado que tanto el parámetro \textit{mean} como el parámetro $\delta$ son relevantes en cuanto a la detección de anomalías, se plantea combinar ambos métodos para mejorar la detección de anomalías.

\begin{figure}[Distribución de los parámetros $\delta$ del ajuste $\alpha$-estable y \textit{mean}]{FIG:SCATTERPLOTMIXED}{Distribución de los parámetros $\delta$ del ajuste $\alpha$-estable y \textit{mean}}
    \image{8cm}{}{scatter_plot_mixed.pdf}
    \label{FIG:SCATTERPLOTMIXED}
\end{figure}

% Correlation table:
% delta_values  mean_values
% delta_values      1.000000     0.003454
% mean_values       0.003454     1.000000

\begin{table}[Correlación entre los valores de $\delta$ y \textit{mean}]{TAB:CORRELACION}{Correlación entre los valores de $\delta$ y \textit{mean}}
    \begin{tabular}{|c|c|c|}
        \hline
        & $\delta$ & Media \\
        \hline
        $\delta$ & 1.000000 & 0.003454 \\
        Media & 0.003454 & 1.000000 \\
        \hline
    \end{tabular}
\end{table}

En la figura \ref{FIG:SCATTERPLOTMIXED} se muestra la distribución de los valores de $\delta$ y \textit{mean} obtenidos en los apartados anteriores. Se ha calculado la matrix de correlación entre ambos parámetros, que se muestra en la tabla \ref{TAB:CORRELACION}. Se observa que la correlación entre ambos parámetros es muy baja, por lo que se plantea que la combinación de ambos métodos puede ser útil para la detección de anomalías.