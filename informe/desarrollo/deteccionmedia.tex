De forma similar a la detección de anomalías mediante la comparación de distribuciones $\alpha$-estables, dada una ventana de 15 minutos, se han tipificado los datos de la subventana de 3 minutos en la que se quiere detectar un posible ataque. Dicho proceso de tipificación se ha realizado con la media y desviación estándar de la subventana de 12 minutos tomada como referencia, como se indica en la sección \ref{SUBSEC:NORMALIZACIONTIPIFICACION}. Posteriormente, se ha calculado la media y desviación estándar de los datos tipificados de la subventana de 3 minutos. Para evitar confusión en la lectura, estos valores de media y desviación estándar se referirán como parámetros \textit{mean} y \textit{std} respectivamente.

\begin{figure}[Distribución de los parámetros \textit{mean} y \textit{std} tras la tipificación]{FIG:SCATTERPLOTTIPIFICACION}{Distribución de los parámetros \textit{mean} y \textit{std} tras la tipificación}
    \image{8cm}{}{scatter_plot_mean.pdf}
    \label{FIG:SCATTERPLOTTIPIFICACION}
\end{figure}

En la figura \ref{FIG:SCATTERPLOTTIPIFICACION} se muestra la distribución de los parámetros \textit{mean} y \textit{std} obtenidos en la tipificación de las subventanas de 3 minutos. En color rojo indican que la ventana estaba sujeta a un ataque, mientras que en color azul se indica que no lo estaba. 
Al igual que en la figura \ref{FIG:SCATTERPLOTALFA}, en diagonal observamos los histogramas de los parámetros, donde se muestra la distribución que siguen los valores obtenidos según estuvieran bajo ataque.

\begin{table}[Características de la distribución del parámetro \textit{mean}]{TAB:CARACMEDIA}{Características de la distribución del parámetro \textit{mean}}
    \begin{tabular}{|c|c|c|c|}
        \hline
        & Media & Desviación estándar \\
        \hline
        Ataque & $0.0339$ & $0.0270$ \\
        Sin ataque & $0.0005$ & $0.0235$ \\
        \hline
    \end{tabular}
\end{table}

En el histograma de \textit{mean} se observa una clara diferenciación entre las distribuciones de los valores obtenidos cuando hay un ataque y cuando no lo hay; en la localización de la distribución.
Como vemos en la tabla \ref{TAB:CARACMEDIA}, la distribución de los valores de \textit{mean} cuando no hay ataque se encuentra en torno a $0$, mientras que cuando hay un ataque se encuentra en torno a $0.34$. Además, la desviación estándar de los valores de \textit{mean} es relativamente pequeña, lo que nos permite diferenciar entre ambas distribuciones.

Por el contrario, en el histograma de \textit{std} no se observa ninguna diferencia entre las distribuciones que siguen estos valores. En el diagrama de dispersión se observa únicamente un desplazamiento de los puntos en el eje correspondiente a \textit{mean}, lo que indica que \textit{std} no es útil para la detección de anomalías.
Por tanto, parece razonable considerar únicamente el parámetro \textit{mean} para la detección de anomalías.