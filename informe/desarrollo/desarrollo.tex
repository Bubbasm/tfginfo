En esta sección se describen las pruebas realizadas para evaluar el rendimiento del sistema de detección de anomalías. Se aplicarán las pruebas descritas en el capítulo \ref{CAP:SISTEMA} para todas las ventanas posibles. Las ventanas que se consideran tienen un tamaño de 15 minutos y se desplazan cada 1 minuto a lo largo de la serie temporal, lo que permite una evaluación detallada del sistema de detección. Sin embargo, el tipo de ataque se ha limitado a un ataque de tráfico incremental, ya que es el más difícil de percibir y, por tanto, el más interesante de detectar. Un ataque con un aumento de tráfico en pico es más fácil de detectar, ya que se puede observar un aumento brusco en el tráfico, como vimos en la sección \ref{SEC:ESTACIONARIEDAD}, en la figura \ref{FIG:ESTACIONARIEDADATAQUE}.
Además, el tipo de diferenciación que se aplica es mediante convoluciones, ya que extienden los efectos de los ataques en la serie temporal.

En las pruebas se analizarán los falsos positivos y falsos negativos obtenidos, para determinar la eficacia del sistema.
También se aplicarán métodos para mejorar las predicciones, como regresión logística y \ac{SVM}.