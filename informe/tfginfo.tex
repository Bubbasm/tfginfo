\documentclass[epsbased,copyright,final,printable,covers,extendedindex,firstnumbered,tfg,gnuplot]{tfgtfmthesisuam}

\usepackage{svg}

\advisor{Jorge Enrique López de Vergara Méndez}
\levelin{Ingeniería Informática}
\copyrightdate{2024}
\title{Aplicación de técnicas de análisis de series temporales a tráfico de red}
\author{Bhavuk Sikka Bajaj}

\dedication{A mis padres, por su apoyo incondicional}
\famouscite{The goal of forecasting is not to predict the future \\but to tell you what you need to know to take meaningful action in the present \\[0.1em] \begin{flushright} Paul Saffo \end{flushright}}
% \prefacefile{inicio/prefacio}
\ackfile{inicio/agradecimientos}
\resumenfile{inicio/resumen}
\abstractfile{inicio/abstract}
\coverdata
{
  Escuela Politécnica Superior \\
  Universidad Autónoma de Madrid \\
  C\textbackslash Francisco Tomás y Valiente nº 11
}

\bibliographyconfig{tfginfo}

\datadir{data}
\graphicsdir{img}
\logosdir{logos}
\codesdir{codes}

\begin{document}

\newacronym{IDS}{IDS}{Sistema de Detección de Intrusiones, \textit{Intrusion Detection System}}
\newacronym{DOS}{DoS}{Denegación de Servicio, \textit{Denial of Service}}
\newacronym{DDOS}{DDoS}{Denegación de Servicio Distribuidos, \textit{Distributed Denial of Service}}
\newacronym{TFG}{TFG}{Trabajo de Fin de Grado}
\newacronym{TFM}{TFM}{Trabajo de Fin de Máster}
\newacronym{UGR}{UGR}{Universidad de Granada}
\newacronym{UNB}{UNB}{Universidad de New Brunswick}
\newacronym{CIC}{CIC}{Instituto de Ciberseguridad de Canadá, \textit{Canadian Institute for Cybersecurity}}
\newacronym{SVM}{SVM}{Máquinas de Vectores de Soporte, \textit{Support Vector Machine}}
\newacronym{UDP}{UDP}{Protocolo de Datagramas de Usuario, \textit{User Datagram Protocol}}
\newacronym{ICMP}{ICMP}{Protocolo de Mensajes de Control de Internet, \textit{Internet Control Message Protocol}, }
\newacronym{HTTP}{HTTP}{Protocolo de Transferencia de Hipertexto, \textit{Hypertext Transfer Protocol}}
\newacronym{RNN}{RNN}{Redes Neuronales Recurrentes, \textit{Recurrent Neural Networks}}
\newacronym{FDA}{FDA}{Análisis de Datos Funcionales, \textit{Functional Data Analysis}}
\newacronym{ADF}{ADF}{Dickey-Fuller Aumentado, \textit{Augmented Dickey-fuller}}
\newacronym{KPSS}{KPSS}{Kwiatkowski-Phillips-Schmidt-Shin}
\newacronym{K-S}{K-S}{Kolmogorov-Smirnov}
\newacronym{MSE}{MSE}{Error Cuadrático Medio, \textit{Mean Squared Error}}
\newacronym{RBF}{RBF}{Función de Base Radial, \textit{Radial Basis Function}}
\newacronym{ROC}{ROC}{Característica Operativa del Receptor, \textit{Receiving Operating Characteristic}}

\chapter{Introducción\label{CAP:INTRODUCCION}}{introduccion/introduccion}
  \section{Motivación\label{SEC:MOTIVACION}}{introduccion/motivacion}
  \section{Objetivos\label{SEC:OBJETIVOS}}{introduccion/objetivos}
  \section{Fases de realización\label{SEC:REALIZACION}}{introduccion/realizacion}
  \section{Organización de la memoria\label{SEC:ORGANIZACION}}{introduccion/organizacion}

\chapter{Estado del arte\label{CAP:ESTADOARTE}}{estadoarte/estadoarte}
  \section{Ataques en la red\label{SEC:ATAQUES}}{estadoarte/ataques}
  % trabajos previos
  % mencionar simmross y copia https://apps.dtic.mil/sti/trecms/pdf/AD1143584.pdf
  \section{Estudios anteriores de series temporales en tráfico de red\label{SEC:ESTUDIOS}}{estadoarte/estudios} % moverlo arriba
  \section{Datos utilizados en los estudios\label{SEC:DATOS}}{estadoarte/datos}
  \section{Conclusión\label{SEC:EACONCLUSION}}{estadoarte/conclusion}

\chapter{Análisis de datos\label{CAP:ANALISIS}}{analisis/analisis}
  \section{Herramientas de análisis\label{SEC:HERRAMIENTAS}}{analisis/herramientas}
  \section{Descripción de los datos\label{SEC:DESCRIPCION}}{analisis/descripcion}
  \section{Simulación de ataque\label{SEC:SIMULACION}}{analisis/simulacion}
  \section{Conclusión\label{SEC:ADCONCLUSION}}{analisis/conclusion}

\chapter{Diseño del sistema\label{CAP:SISTEMA}}{sistema/sistema}
  \section{Estacionarización de los datos\label{SEC:ESTACIONARIEDAD}}{sistema/estacionariedad}
  \section{Esquema de funcionamiento\label{SEC:ESQUEMA}}{sistema/esquema}
  \section{Conclusión\label{SEC:SDCONCLUSION}}{sistema/conclusion}

\chapter{Resultados obtenidos\label{CAP:DESARROLLO}}{desarrollo/desarrollo}
\section{Detección mediante $\alpha$-estables\label{SEC:DETECCIONALFA}}{desarrollo/deteccionalfa}
  \section{Detección mediante la media\label{SEC:DETECCIONMEDIA}}{desarrollo/deteccionmedia}
  \section{Detección combinada\label{SEC:DETECCION}}{desarrollo/deteccion}
  \section{Métricas de evaluación\label{SEC:METRICAS}}{desarrollo/metricas}
  \section{Conclusión\label{SEC:DPCONCLUSION}}{desarrollo/conclusion}
  % hacer plot de que ataques se detectan y cuales no
\chapter{Conclusiones\label{CAP:CONCLUSIONES}}{conclusiones/conclusiones}
  \section{Trabajos futuros\label{SEC:FUTUROS}}{conclusiones/futuros}

\appendix

\chapter{Enlace a código\label{CAP:ANEXOA}}{anexos/anexoA}


\end{document}

