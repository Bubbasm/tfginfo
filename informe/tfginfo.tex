\documentclass[epsbased,copyright,final,printable,covers,extendedindex,firstnumbered,tfg,gnuplot]{tfgtfmthesisuam}

\advisor{Jorge Enrique López de Vergara Méndez}
\levelin{Ingeniería Informática}
\copyrightdate{2020}
\title{Aplicación de técnicas de análisis de series temporales a tráfico de red}
\author{Bhavuk Sikka Bajaj}

\dedication{A mis padres, por su apoyo incondicional}
\famouscite{}
\prefacefile{inicio/prefacio}
\ackfile{inicio/agradecimientos}
\resumenfile{inicio/resumen}
\abstractfile{inicio/abstract}
\coverdata
{
  Escuela Politécnica Superior \\
  Universidad Autónoma de Madrid \\
  C\textbackslash Francisco Tomás y Valiente nº 11
}

\bibliographyconfig{tfginfo}

\datadir{data}
\graphicsdir{img}
\logosdir{logos}
\codesdir{codes}

\begin{document}

\chapter{Introducción\label{CAP:INTRODUCCION}}{introduccion/introduccion}
  \section{Motivación\label{SEC:MOTIVACION}}{introduccion/motivacion}
  \section{Objetivos\label{SEC:OBJETIVOS}}{introduccion/objetivos}
  \section{Pase de realización\label{SEC:REALIZACION}}{introduccion/realizacion}
  \section{Organización de la memoria\label{SEC:ORGANIZACION}}{introduccion/organizacion}

\chapter{Estado del arte\label{CAP:ESTADOARTE}}{estadoarte/estadoarte}
  \section{Ataques en la red\label{SEC:ATAQUES}}{estadoarte/ataques}
  \section{Datos utilizados en los estudios\label{SEC:DATOS}}{estadoarte/datos}
  % añadir apartado de alfa-estables y tests adf que se realizan. Tambien una subsubseccion para la diferenciacion
  \section{Series temporales\label{SEC:SERIESTEMPORALES}}{estadoarte/series}
  \section{Conclusion\label{SEC:EACONCLUSION}}{estadoarte/conclusion}

\chapter{Análisis de datos\label{CAP:ANALISIS}}{analisis/analisis}
  \section{Estudios anteriores de series temporales en tráfico de red\label{SEC:ESTUDIOS}}{analisis/estudios}
  \section{Herramientas de análisis\label{SEC:HERRAMIENTAS}}{analisis/herramientas}
  \section{Descripción de los datos\label{SEC:DESCRIPCION}}{analisis/descripcion}
  \section{Simulación de ataque\label{SEC:SIMULACION}}{analisis/simulacion}
  \section{Conclusion\label{SEC:ADCONCLUSION}}{analisis/conclusion}

\chapter{Sistema y diseño\label{CAP:SISTEMA}}{sistema/sistema}
  \section{Esquema de funcionamiento\label{SEC:ESQUEMA}}{sistema/esquema}
  \section{Conclusion\label{SEC:SDCONCLUSION}}{sistema/conclusion}

\chapter{Desarrollo y pruebas\label{CAP:DESARROLLO}}{desarrollo/desarrollo}
  \section{Conclusion\label{SEC:DPCONCLUSION}}{desarrollo/conclusion}

\chapter{Conclusiones\label{CAP:CONCLUSIONES}}{conclusiones/conclusiones}
  \section{Trabajos futuros\label{SEC:FUTUROS}}{conclusiones/futuros}

\appendix

\chapter{Enlace a código\label{CAP:ACRONIMOS}}{anexos/anexoA}


\end{document}

