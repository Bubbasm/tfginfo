\documentclass[epsbased,copyright,final,printable,covers,extendedindex,firstnumbered,tfg,gnuplot]{tfgtfmthesisuam}

\usepackage{svg}

\advisor{Jorge Enrique L\'opez de Vergara M\'endez}
\levelin{Ingenier\'ia Inform\'atica}
\copyrightdate{2024}
\title{Aplicaci\'on de t\'ecnicas de an\'alisis de series temporales a tr\'afico de red}
\author{Bhavuk Sikka Bajaj}

\dedication{A mis padres, por su apoyo incondicional}
\famouscite{The goal of forecasting is not to predict the future \\but to tell you what you need to know to take meaningful action in the present \\[0.1em] \begin{flushright} Paul Saffo \end{flushright}}
% \prefacefile{inicio/prefacio}
\ackfile{inicio/agradecimientos}
\resumenfile{inicio/resumen}
\abstractfile{inicio/abstract}
\coverdata
{
  Escuela Polit\'ecnica Superior \\
  Universidad Aut\'onoma de Madrid \\
  C\textbackslash Francisco Tom\'as y Valiente nº 11
}

\bibliographyconfig{tfginfo}

\datadir{data}
\graphicsdir{img}
\logosdir{logos}
\codesdir{codes}

\begin{document}

\newacronym{IDS}{IDS}{Sistema de Detección de Intrusiones}
\newacronym{DOS}{DoS}{Denegación de Servicio}
\newacronym{DDOS}{DDoS}{Distribuidos de Denegación de Servicio}
\newacronym{TFG}{TFG}{Trabajo de Fin de Grado}
\newacronym{TFM}{TFM}{Trabajo de Fin de Máster}
\newacronym{UGR}{UGR}{Universidad de Granada}
\newacronym{UNB}{UNB}{Universidad de New Brunswick}
\newacronym{CIC}{CIC}{Instituto de Ciberseguridad de Canadá}
\newacronym{SVM}{SVM}{Máquinas de Vectores de Soporte}
\newacronym{UDP}{UDP}{Protocolo de Datagramas de Usuario}
\newacronym{ICMP}{ICMP}{Protocolo de Mensajes de Control de Internet}
\newacronym{HTTP}{HTTP}{Protocolo de Transferencia de Hipertexto}
\newacronym{RNN}{RNN}{Redes Neuronales Recurrentes}
\newacronym{FDA}{FDA}{Análisis de Datos Funcionales}

\chapter{Introducci\'on\label{CAP:INTRODUCCION}}{introduccion/introduccion}
  \section{Motivaci\'on\label{SEC:MOTIVACION}}{introduccion/motivacion}
  \section{Objetivos\label{SEC:OBJETIVOS}}{introduccion/objetivos}
  \section{Fases de realizaci\'on\label{SEC:REALIZACION}}{introduccion/realizacion}
  \section{Organizaci\'on de la memoria\label{SEC:ORGANIZACION}}{introduccion/organizacion}

\chapter{Estado del arte\label{CAP:ESTADOARTE}}{estadoarte/estadoarte}
  \section{Ataques en la red\label{SEC:ATAQUES}}{estadoarte/ataques}
  % trabajos previos
  % mencionar simmross y copia https://apps.dtic.mil/sti/trecms/pdf/AD1143584.pdf
  \section{Estudios anteriores de series temporales en tr\'afico de red\label{SEC:ESTUDIOS}}{estadoarte/estudios} % moverlo arriba
  \section{Datos utilizados en los estudios\label{SEC:DATOS}}{estadoarte/datos}
  \section{Conclusi\'on\label{SEC:EACONCLUSION}}{estadoarte/conclusion}

\chapter{An\'alisis de datos\label{CAP:ANALISIS}}{analisis/analisis}
  \section{Herramientas de an\'alisis\label{SEC:HERRAMIENTAS}}{analisis/herramientas}
  \section{Descripci\'on de los datos\label{SEC:DESCRIPCION}}{analisis/descripcion}
  \section{Simulaci\'on de ataque\label{SEC:SIMULACION}}{analisis/simulacion}
  \section{Conclusi\'on\label{SEC:ADCONCLUSION}}{analisis/conclusion}

\chapter{Dise\~no del sistema\label{CAP:SISTEMA}}{sistema/sistema}
  \section{Estacionarizaci\'on de los datos\label{SEC:ESTACIONARIEDAD}}{sistema/estacionariedad}
  \section{Esquema de funcionamiento\label{SEC:ESQUEMA}}{sistema/esquema}
  \section{Conclusi\'on\label{SEC:SDCONCLUSION}}{sistema/conclusion}

\chapter{Resultados obtenidos\label{CAP:DESARROLLO}}{desarrollo/desarrollo}
\section{Detecci\'on mediante $\alpha$-estables\label{SEC:DETECCIONALFA}}{desarrollo/deteccionalfa}
  \section{Detecci\'on mediante la media\label{SEC:DETECCIONMEDIA}}{desarrollo/deteccionmedia}
  \section{Detecci\'on combinada\label{SEC:DETECCION}}{desarrollo/deteccion}
  \section{M\'etricas de evaluaci\'on\label{SEC:METRICAS}}{desarrollo/metricas}
  \section{Conclusi\'on\label{SEC:DPCONCLUSION}}{desarrollo/conclusion}
  % hacer plot de que ataques se detectan y cuales no
\chapter{Conclusiones\label{CAP:CONCLUSIONES}}{conclusiones/conclusiones}
  \section{Trabajos futuros\label{SEC:FUTUROS}}{conclusiones/futuros}

\appendix

\chapter{Enlace a c\'odigo\label{CAP:ANEXOA}}{anexos/anexoA}


\end{document}

