Las herramientas predominantes en el estudio fueron MATLAB y Python. A continuación, se describen las tareas más importantes realizadas con cada una de ellas:

\begin{itemize}
\item \textbf{MATLAB:} Se empleó para analizar los estudios de Stresklovaya, aunque no se obtuvieron resultados significativos. Además, MATLAB se utilizó para realizar ajustes de datos a distribuciones $\alpha$-estables debido a la complejidad del algoritmo implementado en Python, que requería un tiempo considerable para completar los ajustes.
\item \textbf{Python:} Fue la principal herramienta utilizada para realizar diversas pruebas y análisis en el estudio. Se utilizaron librerías como numpy, pandas, statsmodels y sklearn para el tratamiento de series temporales y otras utilidades. Las pruebas realizadas en Python incluyeron:
\begin{itemize}
    \item Tests de estacionariedad: \ac{ADF} \cite{dickey1979distribution}, \ac{KPSS} \cite{kwiatkowski1992testing} y \ac{K-S} \cite{kolmogorov1933sulla}.
    \item Diferenciación de series que se abordará más adelante en el estudio.
    \item Uso de \textit{wavelets} para el análisis de series temporales.
    \item Descomposición de series utilizando la función \texttt{seasonal\_decompose}.
\end{itemize}
\end{itemize}

