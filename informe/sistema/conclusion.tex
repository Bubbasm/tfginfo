En este capítulo hemos presentado una metodología para la detección de anomalías en el tráfico de red que se basa en la estandarización de los datos. Esta metodología necesita que los datos de entrada sean estacionarios, por lo que hemos aplicado técnicas de diferenciación y convolución para garantizar la estacionariedad de los datos. Hemos encontrado que una diferenciación de primer orden es suficiente para lograr la estacionarización de los datos. Además, hemos explorado una técnica alternativa utilizando una convolución con un \textit{kernel} específico. Ambas técnicas son efectivas para hacer que los datos sean estacionarios, pero la convolución tiene la propiedad de extender los efectos de los ataques en la serie temporal.

Con estos métodos en mente, podemos aplicar la metodología de detección de anomalías sobre los datos estacionarizados. En el siguiente capítulo, analizaremos los resultados obtenidos y evaluaremos la efectividad de nuestra estrategia para detectar y caracterizar anomalías en el tráfico de red.