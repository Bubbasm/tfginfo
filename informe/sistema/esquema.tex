TODO: Introducir que utilizamos la escationarizacion obtenida mediante la diferenciacion de los datos, y ahora aplicamos las pruebas realizadas por Benjamin en su estudio

\subsection{Normalización de los datos}

TODO: También proponemos otro método para comparar los datos de un tráfico de red de una ventana fija, que es la normalización de los datos. 
Hablar sobre que es la normalización de los datos cuando se tienen datos discretos. Hablar de que efectivamente se puede aplicar este método gracias a la estacionariedad de los datos
y que además, como cada ventana se compone de muchos puntos (898), es correcto hacerlo

\subsection{Ajuste $\alpha$-estable}

TODO: hablar que se ha tenido que utiizar MATLAB para realizar el ajuste de los datos a una distribución $\alpha$-estable, mediante el uso de la función \texttt{fitdist}, con parametro \texttt{'Stable'}.

