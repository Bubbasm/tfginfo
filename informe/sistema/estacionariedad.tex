
TODO: Realizar un plot de los datos sin diferenciar ni nada.

\subsection{Diferenciación de los datos}

TODO: hablar sobre que es la diferenciacion discreta, con sus referencias y como la aprovechamos en este caso para hacer que los datos sean estacionarios.
Tambien hablar sobre la diferenciacion de orden superior, pero que en este caso no nos hace falta ya que los datos son estacionarios con una diferenciacion de orden 1.
También hablar sobre dos metodos para la diferenciacion: 
    1. Diferenciación discreta, restando un valor del previo
    2. Diferenciación mediante una convolución, con el kernel [-1/2, 0, 1/2]

TODO: Realizar un plot de los datos diferenciados.

\subsection{Pruebas de estacionariedad}

TODO: Mostrar las tres pruebas que se han realizado para comprobar la estacionariedad de los datos, y como se han aplicado en este caso.
Las pruebas son KPSS, ADF y K-S.
Mostrar tablas con los resultados de las pruebas.