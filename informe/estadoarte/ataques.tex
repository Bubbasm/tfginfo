Un ataque de denegación de servicio busca incapacitar un sistema informático que proporciona servicios a los usuarios mediante el envío masivo de solicitudes que sobrecargan la capacidad de procesamiento del servidor. Como resultado, los usuarios que estén utilizando el servicio en ese momento pueden perder la conexión hasta que la situación se normalice. Estos ataques siguen siendo comunes debido a su facilidad de ejecución y el daño que causan a las entidades proveedoras de servicios. Existen diversas modalidades para llevar a cabo un ciberataque, siendo uno de los más comunes el de \ac{DOS} y su variante distribuida \ac{DDOS}.

Numerosos sistemas de protección contra ataques \ac{DOS} utilizan la dirección IP como criterio para bloquear paquetes provenientes de un nodo identificado como potencial atacante. Esta estrategia puede no ser efectiva para detectar un ataque \ac{DDOS}, ya que este tipo de ataques involucra múltiples direcciones IP. Sin embargo, en este estudio se propone una solución que independiente del método de ataque, empleando un sistema de detección basado en el análisis del comportamiento del ancho de banda durante la ocurrencia de tales ataques.

Algunos de los métodos más comunes para llevar a cabo el ataque son los siguientes~\cite{bhosale2017distributed}:
\begin{itemize}
    \item \textbf{\textit{UDP Flood}}: En este ataque, se envían grandes volúmenes de paquetes del \ac{UDP} a un servidor o red. Como \ac{UDP} es un protocolo sin conexión, el servidor no espera confirmación de recepción, lo que facilita la sobrecarga de recursos.
    
    \item \textbf{\textit{ICMP Flood (Ping Flood)}}: Consiste en enviar una gran cantidad de paquetes del \ac{ICMP} \textit{Echo Request} (también conocido como \textit{ping}) a un sistema, con el objetivo de agotar los recursos de procesamiento del servidor al procesar y responder a cada solicitud.
    
    \item \textbf{\textit{HTTP Flood}}: Este ataque implica enviar un gran volumen de solicitudes del \ac{HTTP} válidas pero maliciosas a un servidor web, con el propósito de agotar los recursos de procesamiento del servidor al atender estas solicitudes.
    
    \item \textbf{\textit{Slowloris}}: En este ataque, el atacante envía múltiples solicitudes \ac{HTTP} incompletas y mantiene las conexiones abiertas el mayor tiempo posible, consumiendo los recursos del servidor al mantener muchas conexiones simultáneas abiertas.
\end{itemize}
En cualquiera de estos casos, observamos un incremento sustancial en el uso del ancho de banda de la red, ya sea en número de bytes o de paquetes, lo que puede ser detectado mediante el análisis de series temporales.
En este trabajo, se explorará cómo detectar esta anomalía.