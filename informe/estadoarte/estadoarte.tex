La investigación sobre detección de anomalías en redes aborda dos áreas cruciales: los tipos de ataques en la red y los estudios previos que emplean series temporales para analizar el tráfico. En este contexto, es esencial comprender los principios de seguridad en Internet y la amenaza que representan los ataques \ac{DOS}. Estos ataques buscan saturar los recursos de red legítimos para interrumpir servicios.

Desde una perspectiva más específica, se enfoca en los ataques \ac{DOS} de reducida tasa, como ejemplo de anomalía sutil que puede pasar desapercibida en métodos de detección convencionales. El análisis de series temporales en el tráfico de red ha sido explorado en estudios anteriores, utilizando técnicas como distribuciones $\alpha$-estables, \textit{B-Splines}, \ac{FDA} y suavizado exponencial de Holt-Winters. Estas metodologías permiten modelar y predecir el comportamiento del tráfico, identificando patrones anómalos que podrían indicar actividades maliciosas.

La investigación se apoya en datos específicos, obtenidos por su calidad y relevancia en trabajos previos.
