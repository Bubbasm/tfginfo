En resumen, el estado del arte abordado en este estudio ofrece una visión detallada de los enfoques y técnicas empleadas en la detección de anomalías en el tráfico de red mediante el análisis de series temporales. Se han revisado aspectos fundamentales como los tipos de ataques en la red, los métodos de modelado y análisis de series temporales, incluyendo técnicas como el ajuste a las distribuciones $\alpha$-estables y el suavizado exponencial de Holt-Winters. También se destaca la importancia de utilizar datasets variados y representativos para validar los métodos propuestos.

Las metodologías introducidas en este capítulo se aplicarán y evaluarán en el siguiente capítulo utilizando el dataset UGR16~\cite{datosugr16}.