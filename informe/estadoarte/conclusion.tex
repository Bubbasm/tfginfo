En resumen, el estado del arte abordado en este estudio proporciona una visión amplia de los diferentes enfoques y técnicas utilizados en la detección de anomalías en el tráfico de red mediante el análisis de series temporales. Se han revisado aspectos fundamentales como los tipos de ataques en la red, los métodos de modelado y análisis de series temporales, incluyendo técnicas como el ajuste a las distribuciones $\alpha$-estables y el suavizado exponencial de Holt-Winters. También destaca la importancia de los datasets utilizados para validar los métodos propuestos.
El uso de datos variados y representativos es fundamental para la detección adecuada de anomalías en tráfico de red.

Las metodologías introducidas en este capítulo se pondrán a prueba en el capítulo siguiente aplicando los datos utilizados en este estudio, correspondientes al dataset UGR16~\cite{datosugr16}.