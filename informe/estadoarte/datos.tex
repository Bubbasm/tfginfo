Para la elección de los datos utilizados para realizar el análisis y las pruebas, se ha dado prioridad a la consideración de \textit{datasets} relevantes para la continuación del trabajo iniciado por Benjamín Martín. En particular, se ha evaluado el \textit{dataset} UGR16 de la  \ac{UGR}~\cite{datosugr16}, utilizado en su investigación, que abarca varios meses de tráfico de red. 

Sin embargo, se han considerado otros conjuntos de datos, como los proporcionados por el \ac{CIC} de la \ac{UNB}~\cite{datosunb19}. Sin embargo, estos últimos no fueron seleccionados debido a la cantidad limitada de datos que contienen, ya que se trata de tráfico recopilado durante pocas horas durante un período de dos días únicamente. Esta elección se basa en la necesidad de disponer de datos suficientes y representativos para analizar y desarrollar métodos efectivos de detección de anomalías en el tráfico de red.

Para asegurar la calidad y relevancia de los datos, se han utilizado registros extensos de tráfico de red, que abarcan períodos de tiempo significativos y contienen una variedad de patrones de comportamiento. Estos datos incluyen mediciones de flujo de red, paquetes capturados en diferentes segmentos de la red y parámetros específicos relacionados con el tráfico. Dado que la información se presenta en forma de flujos, es necesario procesarla para convertirla en datos de bits y paquetes por segundo.
Es importante destacar que Benjamín Martín ya ha realizado esta labor en su \ac{TFG}~\cite{benjamin2021}, lo que facilita la comparación y la continuidad del trabajo.

\begin{figure}[Tráfico completo del dataset UGR16~\cite{datosugr16}]{FIG:DATASET}{Tráfico completo del dataset UGR16~\cite{datosugr16}}
    \image{}{}{UGRtrafico.pdf}
    \label{FIG:DATASET}
\end{figure}

En la figura~\ref{FIG:DATASET} se muestra el tráfico completo del \textit{dataset} UGR16, que se utilizará en el desarrollo de este trabajo. Debido a la presencia de anomalías ya presentes en el \textit{dataset}, correspondiente a tráfico no usual, se consideran ventanas de tiempo reducidas para analizar, de forma que no contengan anomalías conocidas. En particular, los puntos de color rojo muestran picos de tráfico inusual. Se ha decidido tomar esta medida para evitar que las anomalías presentes en el \textit{dataset} afecten el análisis y la detección de anomalías en el tráfico de red.

De esta forma, para la elaboración del estudio se utilizarán dos series temporales: 
\begin{itemize}
    \item \textbf{Tráfico normal}: Representa el tráfico normal sin ataques. Los datos en esta serie incluyen la tasa binaria o \textit{bitrate} y la tasa de paquetes o \textit{packet rate} del tráfico de red, medidos en bits y paquetes por segundo, respectivamente. Cada punto en la serie corresponde a una medición por segundo.
    \item \textbf{Tráfico de ataque}: Corresponde al tráfico de ataque, generado a partir del tráfico normal mediante la inyección de tráfico malicioso. Al igual que el tráfico normal, incluye mediciones de \textit{bitrate} y \textit{packet rate}, con una frecuencia de medición de un punto por segundo.
\end{itemize}

