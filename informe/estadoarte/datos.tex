Para la elección de los datos utilizados para realizar el análisis y las pruebas, se ha dado prioridad a la consideración de datasets relevantes para la continuación del trabajo iniciado por Benjamin. En particular, se ha evaluado el dataset utilizado en su investigación \cite{datosugr16}. 

Sin embargo, se han considerado otros conjuntos de datos, como los proporcionados por el \ac{CIC} de \ac{UNB} \cite{datosunb19}. Sin embargo, estos últimos no fueron seleccionados debido a la cantidad limitada de datos que contienen, ya que se trata de tráfico recopilado durante pocas horas durante un período de dos días únicamente. Esta elección se basa en la necesidad de disponer de datos suficientes y representativos para analizar y desarrollar métodos efectivos de detección de anomalías en el tráfico de red.

Para asegurar la calidad y relevancia de los datos, se han utilizado registros extensos de tráfico de red, que abarcan períodos de tiempo significativos y contienen una variedad de patrones de comportamiento. Estos datos incluyen mediciones de flujo de red, paquetes capturados en diferentes segmentos de la red y parámetros específicos relacionados con el tráfico. Dado que la información se presenta en forma de flujos, es necesario procesarñla para convertirla en datos de bits y paquetes por segundo.
Es importante destacar que Benjamin ya ha realizado esta labor en su \ac{TFG} \cite{benjamin2021}, lo que facilita la comparación y la continuidad del trabajo.

\begin{figure}[Tráfico completo del dataset UGR]{FIG:DATASET}{Tráfico completo del dataset UGR}
    \image{}{}{UGRtrafico.pdf}
    \label{FIG:DATASET}
\end{figure}

En la figura \ref{FIG:DATASET} se muestra el tráfico completo del dataset UGR, que se utilizará en el desarrollo de este trabajo. Debido a la presencia de anomalías ya presentes en el dataset, correspondiente a tráfico no usual, se consideran ventanas de tiempo reducidas para analizar, tal que no contengan anomalías. En particular, los puntos de color rojo muestran picos tráfico inusual. Se ha decidido tomar esta medida para evitar que las anomalías presentes en el dataset afecten el análisis y la detección de anomalías en el tráfico de red.

De esta forma, para la elaboración del estudio se utilizarán dos series temporales: 
\begin{itemize}
    \item \textbf{Tráfico normal}: Representa el tráfico normal sin ataques. Los datos en esta serie incluyen el \textit{bitrate} y el \textit{packetrate} del tráfico de red, medidos en bits y paquetes por segundo, respectivamente. Cada punto en la serie corresponde a una medición por segundo.
    \item \textbf{Tráfico ataque}: Corresponde al tráfico de ataque, generado a partir de la Serie 1 mediante la inyección de tráfico malicioso. Al igual que la Serie 1, incluye mediciones de \textit{bitrate} y \textit{packetrate}, con una frecuencia de medición de un punto por segundo.
\end{itemize}
Ambas series contienen los datos del \textit{bitrate} y el \textit{packetrate} del tráfico de red, medidos en bits y paquetes por segundo, respectivamente. 
