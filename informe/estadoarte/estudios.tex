% TODO: comparar los trabajos de benjamin2023, strelkovskaya2019, cabornero2021, mateo2013, muelas2015
Para poder realizar una investigación de valor, es necesario presentar los estudios previos que han abordado la detección de anomalías en tráfico de red mediante el uso de series temporales. Como continuación del \ac{TFM} de Benjamín Martín, esta investigación utiliza trabajos relacionados que han sido de gran interés para comprender y avanzar en este campo. Además, durante la realización de este trabajo se han encontrado otros estudios que han abordado la detección de anomalías en tráfico de red utilizando diferentes enfoques y técnicas. A continuación, se presentan algunos de estos estudios y se discuten sus contribuciones y hallazgos.

\subsection{$\alpha$-estables} % benjamin2023, mateo2013
La utilización de distribuciones $\alpha$-estables~\cite{nolan2020univariate} ha sido un enfoque importante en la detección de anomalías en el tráfico de red, como se discute en los trabajos de Benjamín Martín~\cite{benjamin2023}, Mateo Stoppa~\cite{mateo2013}, y Federico Simmross~\cite{simmross2011anomaly}.

F. Simmross es reconocido como pionero en el uso de distribuciones $\alpha$-estables para la detección de anomalías en el tráfico de red. Propuso un método novedoso basado en un modelo $\alpha$-estable de primer orden no restringido y pruebas estadísticas de hipótesis. El estudio proporciona evidencia de que la distribución marginal del tráfico real se modela adecuadamente con funciones $\alpha$-estables y clasifica patrones de tráfico mediante una \ac{GLRT}. Se enfoca en detectar dos tipos de anomalías: inundaciones (\textit{floods}) y aglomeraciones repentinas (\textit{flash-crowds}).

El trabajo de M. Stoppa también se centra en el modelado del tráfico de red utilizando distribuciones $\alpha$-estables. Compara parámetros obtenidos de datos de tráfico recopilados a través de protocolos como SNMP y NetFlow para entender mejor las características estadísticas del tráfico y su capacidad para ajustarse a estas distribuciones.

B. Martín propone una técnica de detección de anomalías sutiles que combina modelos predictivos de series temporales basados en \ac{RNN} con estadísticos. Su estudio utiliza una distribución $\alpha$-estable como referencia para modelar el comportamiento usual del tráfico, de modo que la detección de anomalías se basa en identificar desviaciones significativas de este modelo esperado.

Estos estudios subrayan la importancia y utilidad de las distribuciones $\alpha$-estables en la modelización y detección de anomalías en el tráfico de red. Además, proporcionan un marco teórico robusto para caracterizar el comportamiento del tráfico y permiten identificar anomalías mediante la comparación con un modelo de referencia. Sin embargo, el trabajo de M. Stoppa se limita al modelado del tráfico SNMP y NetFlow, y el enfoque de B. Martín con \ac{RNN} incrementa considerablemente el coste computacional de su solución. El trabajo presente busca realizar predicciones de anomalías en la red limitando la complejidad del problema.

\subsection{\textit{B-Splines}} % strelkovskaya2019
Una \textit{B-Spline} (\textit{Basis Spline})~\cite{prautzsch2002bezier} es una función matemática utilizada para aproximar y suavizar datos mediante una combinación lineal de funciones \textit{spline} base locales. Cada función \textit{spline} base es un polinomio definido en un intervalo específico y se utiliza para definir segmentos suaves de una curva. La \textit{B-Spline} completa se forma al combinar estas funciones \textit{spline} base de manera suave y continua, proporcionando una representación flexible de datos y curvas.

En el estudio de Strelkovskaya \textit{et al.}~\cite{strelkovskaya2019}, se aborda la predicción del tráfico autosemejante en redes móviles mediante el uso de diversas funciones \textit{spline}, incluyendo lineales, cúbicas y \textit{B-Splines} cúbicas. Estas funciones \textit{spline} se emplean para predecir el tráfico autosemejante fuera de los períodos de transmisión de datos, aprovechando la capacidad de las \textit{B-Splines} para modelar de manera flexible y suave los patrones complejos presentes en el tráfico de red.

Particularmente, Strelkovskaya \textit{et al.} resaltan la eficacia de las \textit{B-Splines} cúbicas en mejorar la precisión de la predicción del tráfico autosemejante. Al utilizar \textit{B-Splines}, su estudio demuestra que es posible realizar pronósticos precisos tanto a corto como a largo plazo, lo que resulta relevante para la gestión eficiente de las redes móviles y la detección anticipada de posibles anomalías en el tráfico.

\subsection{An\'alisis de Datos Funcionales} % muelas2015
El \ac{FDA} es una técnica utilizada para estudiar y analizar conjuntos de datos representados como funciones, donde cada observación representa una función suave definida en un dominio continuo, como el tiempo. Permite tratar las observaciones como muestras de funciones, facilitando la aplicación de métodos estadísticos y matemáticos para caracterizar y entender la variabilidad en estas funciones.

En el trabajo de Muelas~\cite{muelas2015}, se empleó FDA para analizar patrones de comportamiento de redes a lo largo del tiempo, tratando las medidas de red como funciones continuas y aplicando técnicas específicas para caracterizar y comprender la dinámica de la actividad de red. Esto ha sido parte integral del enfoque para reducir la dimensionalidad de los datos y detectar anomalías en el tráfico de red.

\subsection{Suavizado exponencial de Holt-Winters} % cabornero2021
El suavizado exponencial de Holt-Winters~\cite{chatfield1978holt} es una extensión del método de suavizado exponencial básico que incorpora componentes de tendencia y estacionalidad. Este método se utiliza comúnmente para series de tiempo que muestran variaciones estacionales regulares, como datos de tráfico de red, así como datos económicos mensuales o trimestrales, entre otros.

Las principales características del suavizado exponencial de Holt-Winters incluyen:
\begin{itemize}
    \item \textbf{Componente de Nivel (\textit{Level})}: Representa el nivel base o promedio de la serie temporal.
    \item \textbf{Componente de Tendencia (\textit{Trend})}: Captura la dirección y la tasa de cambio de la serie temporal a lo largo del tiempo.
    \item \textbf{Componente de Estacionalidad (\textit{Seasonality})}: Modela las fluctuaciones periódicas o estacionales en la serie temporal.
\end{itemize}
El método de Holt-Winters utiliza fórmulas de suavizado exponencial para actualizar y estimar estos componentes en cada período de tiempo, permitiendo realizar pronósticos futuros basados en la información histórica de la serie temporal. 

En el trabajo de Cabornero~\cite{cabornero2021}, Holt-Winters se utiliza como técnica de pronóstico para modelar y predecir series temporales de tráfico de red que exhiben patrones. Esto permite analizar y anticipar patrones estacionales, e incluso tendencias en el comportamiento del tráfico.

Además, el método de Holt-Winters se puede ajustar según tres parámetros de suavizado: $\alpha$ para la componente de nivel, $\beta$ para la componente de tendencia y $\gamma$ para la componente de estacionalidad. Estos parámetros controlan la influencia de las observaciones pasadas y la velocidad de adaptación a los cambios en los componentes de la serie temporal. La selección óptima de estos parámetros puede compararse con el ajuste de parámetros en otros algoritmos de optimización, como el recocido simulado (\textit{simulated annealing}). En este caso buscamos la mejor configuración de los parámetros $\alpha$, $\beta$ y $\gamma$ para minimizar el error en las predicciones de la serie temporal.

\subsubsection{\textit{Simulated Annealing}}
\textit{Simulated Annealing}, o recocido simulado en Español, es una técnica de optimización probabilística utilizada para encontrar soluciones aproximadas a problemas de optimización combinatoria~\cite{kirkpatrick1983optimization}.
En nuestro caso, este algoritmo se ha adaptado para ajustar los parámetros del suavizado exponencial de Holt-Winters.