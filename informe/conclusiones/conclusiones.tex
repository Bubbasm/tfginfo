En este trabajo se ha desarrollado un sistema de detección de anomalías en el tráfico de red, como continuación de la labor de Benjamín en su estudio sobre la predictibilidad del tráfico \cite{benjamin2023}. Para ello, se ha obtenido una serie estacionaria a partir de métodos de diferenciación discreta y convoluciones.
Posteriormente, se ha aplicado una metodología basada en la estandarización de los datos, que nos ha permitido comparar los datos de las series temporales. En concreto, se ha estudiado la distribución de los parámetros de la distribución $\alpha$-estable de la serie diferneciada y también su tipificación. Se ha encontrado que el parámetro \textit{mean} de los datos tipificados es un buen indicador para la detección de anomalías, y que la combinación del parámetro \textit{mean} y $\delta$ de la distribución $\alpha$-estable puede mejorar la detección de anomalías.

Se ha concluido que la detección de anomalías es adecuada, aunque la cantidad de falsos positivos y falsos negativos es elevada. Sin embargo, debido a la simplicidad del proceso y poco coste computacional, se considera que es un buen sistema de detección de anomalías.