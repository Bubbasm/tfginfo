% Hablar sobre cómo el proceso descrito en este trabajo puede ayudar a mejorar la detección de ataques en la red.
% Particularmente, que se puede utilizar como una herramienta adicional en un IDS para detectar anomalías en el tráfico.
% El proceso descrito en este trabajo puede contribuir significativamente a mejorar la detección de ataques en la red. 
% Específicamente, puede ser utilizado como una herramienta adicional en un \ac{IDS} 
% para identificar y analizar anomalías en el tráfico.

% La predicción del tráfico proporciona una visión anticipada de los patrones normales de comportamiento de la red. 
% Al comparar estas predicciones con el tráfico en tiempo real, se pueden identificar desviaciones significativas 
% que podrían indicar actividades sospechosas o ataques en curso. Por ejemplo, cambios repentinos en el volumen de datos,
% patrones de comunicación inusuales o flujos de tráfico desconocidos podrían ser señales de posibles intrusiones.

% Al integrar la predicción del tráfico como una capa adicional en un \ac{IDS}, se mejora la capacidad del sistema para 
% detectar y responder a amenazas de manera más proactiva. Esto permite una defensa más robusta contra una amplia gama
% de ataques, proporcionando a los administradores de red una herramienta valiosa para mantener la integridad 
% y la seguridad de la infraestructura de red.

1.- Diferencias de las muestras permiten borrar la estacionalidad
2.- Los datos estadisticos de las series temporales (la media, parametros de la alpha estable), permiten detectar tráfico anómalo