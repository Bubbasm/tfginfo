En este trabajo se ha desarrollado un sistema de detección de anomalías en el tráfico de red, como continuación de la labor de Benjamín Martín en su estudio sobre la predictibilidad del tráfico~\cite{benjamin2023}. Como estudiante del Doble Grado en Ingeniería Informática y Matemáticas, he podido aprovechar los conocimientos adquiridos durante la carrera para abordar el análisis de las series temporales en el contexto del sistema de detección de anomalías en el tráfico de red. Particularmente, el concimiento adquirido en las asignaturas de Redes de Comunicaciones y Estadística han sido de gran utilidad para el desarrollo de este trabajo.

Con el fin de mejorar el sistema inicial propuesto por B. Martín, se ha obtenido una serie estacionaria a partir de métodos de diferenciación discreta y convoluciones.
Posteriormente, se ha aplicado una metodología basada en la estandarización de los datos, que nos ha permitido comparar los datos de las series temporales. En concreto, se ha estudiado la distribución de los parámetros de la distribución $\alpha$-estable de la serie diferenciada y también su tipificación, bajo la hipótesis de que la serie diferenciada sigue un proceso estacionario. Se ha encontrado que el parámetro \textit{mean} de los datos tipificados es un buen indicador para la detección de anomalías, y que la combinación del parámetro \textit{mean} y $\delta$ de la distribución $\alpha$-estable puede mejorar la detección de anomalías. Además, al emplear todos los parámetros combinados (los cuatro parámetros de ajuste $\alpha$-estable: $\alpha$, $\beta$, $\gamma$ y $\delta$; y los dos parámetros obtenidos mediante la tipificación: \textit{mean} y \textit{std}) en una regresión logística, se observa una ligera mejora de las predicciones de anomalías, reduciendo tanto los falsos positivos como los falsos negativos.

Cabe destacar que en este trabajo se aplican bases matemáticas sólidas, como la diferenciación discreta y mediante convoluciones, que nos proporciona una serie estacionaria. Se han utilizado propiedades de dicha serie estacionaria para poder comparar los datos de las series temporales y detectar anomalías.

Se ha concluido que la detección de anomalías es adecuada, aunque la cantidad de falsos positivos y falsos negativos puede ser elevada para un centro de operaciones de ciberseguridad. Sin embargo, debido a la simplicidad del proceso y poco coste computacional, se considera que es un buen sistema de detección temprana de anomalías.