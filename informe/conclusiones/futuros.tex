A pesar de las ventajas significativas del sistema propuesto para la detección de anomalías en el tráfico, como su enfoque genérico y la simplicidad en la predicción de anomalías, existen diversas oportunidades para continuar desarrollándolo y refinándolo. Estas mejoras pueden abordar desafíos específicos y aumentar la efectividad del sistema en entornos de ciberseguridad más complejos y dinámicos. A continuación se detallan algunas posibles mejoras:
\begin{itemize}
    \item \textbf{Ajustar la sensibilidad de la detección}: Actualmente, el número de falsos positivos y falsos negativos aplicando la regresión logística sobre todos los parámetros es similar, entre $15\%$ y $20\%$. Aunque el sistema es adecuado para detectar anomalías, puede ser de interés ajustar la sensibilidad de la detección para reducir el número de falsos positivos y falsos negativos.
    \item \textbf{Mejorar la detección de anomalías}: Se ha encontrado que la combinación de los parámetros del ajuste $\alpha$-estable y los parámetros obtenidos por la tipificación de los datos es eficaz para la detección de anomalías. Sin embargo, se puede investigar el uso de predicciones previas para mejorar la detección de anomalías. Como se ha visto en el capítulo~\ref{CAP:DESARROLLO}, en el apartado~\ref{SUBSEC:DETECCIONDATASET}, el número de falsos positivos no parecían concentrarse en algunos rangos, al contrario que los falsos negativos. Por tanto, se puede investigar sobre un método de predicción que tenga en cuenta múltiples ventanas temporales adyacentes para mejorar la detección de anomalías.
    \item \textbf{Búsqueda de otros parámetros}: En este trabajo se han utilizado los parámetros de la distribución $\alpha$-estable y los parámetros \textit{mean} y \textit{std} obtenidos mediante la tipificación de los datos. Sin embargo, se pueden investigar otros parámetros que puedan mejorar la detección de anomalías. Por ejemplo, se plantea investigar el uso de los cuantiles de la serie diferenciada, con el fin de buscar valores extremos que puedan indicar la presencia de anomalías.
\end{itemize}