A pesar de las ventajas significativas del sistema propuesto para detectar el tráfico, como su enfoque genérico y simplicidad en la predicción de anomalías, hay muchas oportunidades para seguir desarrollándolo, como se detallan a continuación con varias posibles mejoras:
\begin{itemize}
    \item \textbf{Ajustar la sensibilidad de la detección}: Actualmente, el número de falsos positivos y falsos negativos es similar, sobre el $20\%$. Aunque el sistema es adecuado para detectar anomalías, puede ser de interés ajustar la sensibilidad de la detección para reducir el número de falsos positivos o falsos negativos.
    \item \textbf{Mejorar la detección de anomalías}: Se ha encontrado que la combinación de los parámetros del ajuste $\alpha$-estable y los parámetros obtenidos por la tipificación de los datos es eficaz para la detección de anomalías. Sin embargo, se puede investigar el uso de predicciones previas para mejorar la detección de anomalías. Como se ha visto en el capítulo \ref{CAP:DESARROLLO}, en el apartado \ref{SUBSEC:DETECCIONDATASET}, el número de falsos positivos no parecían concentrarse en algunos rangos, al contrario que los falsos negativos. Por tanto, se puede investigar sobre un método de predicción que tenga en cuenta múltiples ventanas temporales adyacentes para mejorar la detección de anomalías.
\end{itemize}